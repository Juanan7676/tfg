\documentclass{beamer}

\mode<presentation> {
  \usetheme{Warsaw}
  \setbeamercovered{transparent}
}

\usepackage[spanish]{babel}
\usepackage[latin1]{inputenc}

\usepackage{amsmath,amssymb}
\usepackage{graphicx}
\usepackage{fancyvrb}

\title{Ejemplo de presentaciones en \textbf{Beamer}}

\author{Asignatura de \LaTeX}

\date{Curso 2006-2007} %

\subject{Generaci�n de presentaciones}

\pgfdeclareimage[height=0.8cm]{SelloUCA.jpg}{logo}
%\logo{\includegraphics[width=8mm]{SelloUCA.jpg}}



%\beamerdefaultoverlayspecification{<+->}

\begin{document}

\begin{frame}
  \titlepage
\end{frame}


\begin{frame}
  \frametitle{Contenido}
  \tableofcontents[pausesections]
\end{frame}


\section{Primera secci�n}

\subsection{Primera subsecci�n}

\begin{frame}
    \frametitle{T�tulo de la primera diapositiva}
\begin{columns}
 \begin{column}{0.55\textwidth}
\begin{block}{Resultados}
    \begin{itemize}[<+->]
      \item  Primer item
      \item  Segundo item
      \item  Tercer item
    \end{itemize}
    \begin{enumerate}[<+-| alert@+>]
      \item  Primer item
      \item  Segundo item
      \item  Tercer item
    \end{enumerate}
\end{block}
 \end{column} \ \
% \begin{column}{0.40\textwidth}
%      \only<4>{\includegraphics[width=0.8\textwidth]{knuth.jpg}}
%      \only<5>{\includegraphics[width=0.9\textwidth]{forges.jpg}}
%      \only<6>{\includegraphics[width=0.9\textwidth]{kill-bill-2.jpg}}
% \end{column}
\end{columns}
\end{frame}


\subsection{Segunda subsecci�n}

\begin{frame}
    \frametitle{T�tulo de la segunda diapositiva}
\begin{block}{}
Escribimos una peque�a f�rmula
\end{block}

\begin{block}{F�rmula}
\begin{equation*}
\onslide<2->{V(x) = } \onslide<3->{\color<3>[rgb]{1,0,0} A\int_0^\infty\frac{dr}{r} +}
\onslide<4->{\color<4>[rgb]{0,1,0} B\int_0^\infty\frac{dr}{r^2} +}
\onslide<5->{\color<5>[rgb]{0,0,1} C\int_0^\infty\left(\frac{1}{r^6} - \frac{1}{r^{12}}\right)dx}
\end{equation*}
\end{block}
\end{frame}

\begin{frame}
    \frametitle{T�tulo de la tercera diapositiva}
\begin{block}{}
Ahora mostramos la f�rmula de forma un poco diferente
\end{block}

\begin{block}{F�rmula}
\begin{equation*}
V(x) =  {\color<2>[rgb]{1,0,0} A\int_0^\infty\frac{dr}{r}} +
{\color<3>[rgb]{1,0,0} B\int_0^\infty\frac{dr}{r^2}} +
{\color<4>[rgb]{1,0,0} C\int_0^\infty\left(\frac{1}{r^6} - \frac{1}{r^{12}}\right)dx}
\end{equation*}
\end{block}
\end{frame}

\begin{frame}
    \frametitle{T�tulo de la tercera diapositiva}
\begin{block}{}
Otra forma un poco m�s compleja:
\end{block}
\begin{block}{Ecuaci�n}
$\displaystyle
V(x) = \onslide<2->{\color<2>[rgb]{1,0,0} A\int_0^\infty\frac{dr}{r} +}
\onslide<3->{\color<3>[rgb]{0,1,0} B\int_0^\infty\frac{dr}{r^2} +}
\onslide<4->{\color<4>[rgb]{0,0,1} C\int_0^\infty\left(\frac{1}{r^6} - \frac{1}{r^{12}}\right)dx}$

\medskip
\hspace*{15mm} \onslide<2>{\color<2>[rgb]{1,0,0} Dipolo} \hspace{7mm}
\onslide<3>{\color<3>[rgb]{0,1,0} Coulomb} \hspace{4mm}
\onslide<4>{\color<4>[rgb]{0,0,1} Van der Waals}
\end{block}
\end{frame}

\begin{frame}[t]
    \frametitle{T�tulo de la tercera diapositiva}
\begin{block}{}
Otra forma reemplazando elementos:
\end{block}
\only<2>{\visible<2>{\begin{block}{Ecuaci�n 1}
$\displaystyle V(x) = A\int_0^\infty\frac{dr}{r} \color[rgb]{1,0,0} \longrightarrow \text{Dipolo}$
\end{block}}}%
\only<3->{\visible<3->{\begin{block}{Ecuaci�n 2}
$\displaystyle W(x) = B\int_0^\infty\frac{dr}{r^2} +
C\int_0^\infty\left(\frac{1}{r^6} - \frac{1}{r^{12}}\right)dx \only<3>{ \longrightarrow \color[rgb]{1,0,0} \text{Coulomb + WdW}}$
\end{block}}}
\only<4>{\visible<4>{\begin{block}{otro bloque}
Otras cosas...
\end{block}}}
\end{frame}




\end{document}

\usetheme{default}
\usetheme{JuanLesPins}
\usetheme{Goettingen}
\usetheme{Szeged}
\usetheme{Warsaw}

\usecolortheme{crane}

\usefonttheme{serif}
\usefonttheme{structuresmallcapsserif}
