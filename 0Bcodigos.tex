\chapter{C�digo de los programas}

\section{Implementaci�n del algoritmo de Shor en Mathematica}

Aqu� presentamos las principales dos funciones implementadas en Mathematica, que realizan las operaciones de las puertas $U_{f_{a,N}}$ y $QFT$ del algoritmo de Shor, estudiado en la secci�n \ref{sec:circuitorden}. El resto del algoritmo particularizado para $N=15$ puede consultarse en el archivo \verb|shor.nb|.

\subsection{Puerta $U_f$}
\begin{lstlisting}[language=Mathematica]
    Uf[estado_, a_, N_, m_, n_] := Module[{final = 0},
    For[i = 0, i <= 2^m - 1, i++,
        For[j = 0, j <= 2^n - 1, j++,
            final += Coefficient[estado, TensorProduct[Ket[i], Ket[j]]]*TensorProduct[Ket[i], Ket[Mod[j + PowerMod[a, i, N], 2^n]]];
        ];
    ];
    Return[final];
];
\end{lstlisting}


\subsection{Puerta $QFT$}
\begin{lstlisting}[language=Mathematica]
QFT[estado_, m_] := Module[{final = 0},
    For[j = 0, j <= 2^m - 1, j++,
        final += 1/Sqrt[2^m]* Coefficient[estado, Ket[j]]*Sum[Exp[(2 \[Pi] I j k )/(2^m)]*Ket[k], {k, 0, 2^m - 1}];
    ];
    Return[final];
];
\end{lstlisting}

\section{Implementaci�n de la firma LD-OTS en Python}

Aqu� presentamos las principales clases y algoritmos implementados en Python usadas en la secci�n \ref{sec:experiment}. Las claves p�blicas y privadas, junto con las firmas generadas se pueden encontrar en la carpeta adjunta. Las claves p�blicas tienen extensi�n \verb|.PUK|, las claves privadas, \verb|.PRK|; y las firmas de los documentos son aquellos ficheros acabados en \verb|.SIG|.

\subsection{Esquema de firma}
Esta es una implementaci�n del sistema de firma de un solo uso de Lamport-Diffie. Las funciones $f$ y $H$ y el par�metro $n$ del esquema se pasan como par�metros. En los experimentos posteriores, hemos usado como $f$ y $H$ la funci�n $SHA-256$ truncada a los primeros \verb|n//8| bytes ($n$ primeros bits si $n$ es m�ltiplo de $8$).
\begin{lstlisting}[language=Python]
class LDOTS:
    # Inicializa la clase. Especificar el numero de bits que devuelve la funcion hash y la funcion de ida (n), la funcion
    # hash y la funcion de ida correspondientes y si queremos verificar usando una clave publica, se pasa como ultimo parametro.
    def __init__(self, n, hash_f, ow_f, pubK = None):
        self.n = n
        self.hash_f = hash_f
        self.ow_f = ow_f
        
        if pubK:
            self.pubK = pubK
            self.privK = None
    
    # Genera un par de claves aleatorias criptograficamente seguras
    def generate_keys(self):
        self.privK = [ ( secrets.randbits(self.n).to_bytes(self.n//8,'big'),secrets.randbits(self.n).to_bytes(self.n//8,'big')) for k in range(self.n) ]
        self.pubK = [ (self.ow_f(k[0]),self.ow_f(k[1])) for k in self.privK ]
    
    # Exporta las claves en los dos ficheros indicados
    def export_keys(self, pubName, privName):
        pubFile = open(pubName+".PUK","wb")
        pubFile.write(pickle.dumps(self.pubK))
        pubFile.close()
        
        privFile = open(privName+".PRK","wb")
        privFile.write(pickle.dumps(self.privK))
        privFile.close()
    
    # Importa la clave publica del fichero especificado 
    def import_pubK(self,name):
        file = open(name,"rb")
        self.pubK = pickle.loads(file.read())
        file.close()
    
    # Importa la clave privada del fichero especificado
    def import_privK(self,name):
        file = open(name,"rb")
        self.privK = pickle.loads(file.read())
        file.close()
    
    # Importa las dos claves del fichero especificado
    def import_keys(self,public,private):
        self.import_pubK(public)
        self.import_privK(private)
    
    # Firma el mensaje especificado. El valor devuelto puede pasarse directamente al metodo verify(). Es necesario contar con la clave privada para usar este metodo.
    def sign(self,msg):
        if self.privK:
            d = bytestobits(self.hash_f(msg))
            return [ self.privK[j][d[j]] for j in range(self.n) ]
        else:
            raise Exception("No hay clave privada; use generate_keys() para generar un par de claves!")
    
    # Devuelve True si el mensaje es autentico, False si no lo es. Es necesario haber inicializado esta clase con la clave publica del emisor.
    def verify(self,msg,signature):
        if self.pubK:
            d = bytestobits(self.hash_f(msg))
            
            for j in range(self.n):
                if self.ow_f(signature[j]) != self.pubK[j][d[j]]: return False
            
            return True
        else:
            raise Exception("No hay clave publica!")
\end{lstlisting}

\subsection{Generador de mensajes}
Para poder generar un conjunto $S$ de mensajes falsificados, normalmente se suele proceder de la siguiente manera: se elabora un texto base para falsificar, y se van insertando sin�nimos entre algunas palabras. As�, eligiendo una palabra de entre cada par de sin�nimos a lo largo del texto, se van obteniendo documentos que son equivalentes, pero con im�genes de la funci�n hash distintas. El documento de ejemplo que se ha usado puede consultarse con m�s detalle en el archivo \verb|crack.py|. Esta clase hace todo el trabajo de ir iterando por las posibles opciones a trav�s de un iterador de Python.

\begin{lstlisting}[language=Python]
class TextoVariable:
    
    def __init__(self,numpos,baseStr):
        self.bstr = baseStr
        self.n = numpos
        self.options = []
        for k in range(numpos):
            self.options.append((2,("","")))
        self.curr = [0 for k in range(self.n)]
    
    def setOption(self,index,options):
        self.options[index] = (len(options),options)
    
    def nextString(self):
        for k in range(self.n):
            if self.curr[k] < (self.options[k][0] - 1):
                self.curr[k] += 1
                break
            else:
                self.curr[k]=0
                if k==self.n - 1:
                    return ""
        
        ret = self.bstr
        for k in range(self.n):
            ret = ret.replace("~", self.options[k][1][self.curr[k]],1)
        return ret

    def iterator(self):
        str = self.nextString()
        while (str!=""):
            yield str
            str = self.nextString()
    def reset(self):
        self.curr = [0 for k in range(self.n)]
\end{lstlisting}

\subsection{Algoritmo de b�squeda por fuerza bruta}
Esta es la implementaci�n del Algoritmo \ref{algfuerzabruta}.
\begin{lstlisting}[language=Python]
start = time.time() # Para medir el tiempo de ejecucion

k=1
for msg in msgGen.iterator(): # msgGen es el generador de mensajes
    digest = lamport.bytestobits(sha256(msg.encode('utf-8')))
    if (digest == dig1):
        print("Documento falsificado!")
        print(msg)
        print("Comprobacion: " + str(l.verify(msg.encode('utf-8'),sig1)))
        break
    k += 1

print(str(k)+" documentos probados.")
print("%s segundos" % (time.time() - start))
\end{lstlisting}

\subsection{Algoritmo de b�squeda con 2 firmas}
Esta es la implementaci�n del Algoritmo \ref{alg2doc}.
\begin{lstlisting}[language=Python]
start = time.time() # Para medir el tiempo de ejecucion

k=1
for msg in msgGen.iterator(): # msgGen es el generador de mensajes
    digest = lamport.bytestobits(sha256(msg.encode('utf-8')))
    mysig = []
    for k in range(len(digest)):
        if digest[k]==dig1[k]:
            mysig.append(sig1[k])
        elif digest[k]==dig2[k]:
            mysig.append(sig2[k])
        else:
            break
    if len(mysig)==len(sig1):
        print("Documento falsificado!")
        print(msg)
        print("Comprobacion: " + str(l.verify(msg.encode('utf-8'),mysig)))
        break
    cont += 1

print(str(cont)+" documentos probados.")
print("%s segundos" % (time.time() - start))
\end{lstlisting}