\chapter{C�digo de los programas}

\section{Implementaci�n de la firma LD-OTS en Python}

\begin{lstlisting}[language=Python]
class LDOTS:
    # Inicializa la clase. Especificar el numero de bits que devuelve la funcion hash y la funcion de ida (n), la funcion
    # hash y la funcion de ida correspondientes y si queremos verificar usando una clave publica, se pasa como ultimo parametro.
    def __init__(self, n, hash_f, ow_f, pubK = None):
        self.n = n
        self.hash_f = hash_f
        self.ow_f = ow_f
        
        if pubK:
            self.pubK = pubK
            self.privK = None
    
    # Genera un par de claves aleatorias criptograficamente seguras
    def generate_keys(self):
        self.privK = [ ( secrets.randbits(self.n).to_bytes(self.n//8,'big'),secrets.randbits(self.n).to_bytes(self.n//8,'big')) for k in range(self.n) ]
        self.pubK = [ (self.ow_f(k[0]),self.ow_f(k[1])) for k in self.privK ]
    
    # Exporta las claves en los dos ficheros indicados
    def export_keys(self, pubName, privName):
        pubFile = open(pubName+".PUK","wb")
        pubFile.write(pickle.dumps(self.pubK))
        pubFile.close()
        
        privFile = open(privName+".PRK","wb")
        privFile.write(pickle.dumps(self.privK))
        privFile.close()
    
    # Importa la clave publica del fichero especificado 
    def import_pubK(self,name):
        file = open(name,"rb")
        self.pubK = pickle.loads(file.read())
        file.close()
    
    # Importa la clave privada del fichero especificado
    def import_privK(self,name):
        file = open(name,"rb")
        self.privK = pickle.loads(file.read())
        file.close()
    
    # Importa las dos claves del fichero especificado
    def import_keys(self,public,private):
        self.import_pubK(public)
        self.import_privK(private)
    
    # Firma el mensaje especificado. El valor devuelto puede pasarse directamente al metodo verify(). Es necesario contar con la clave privada para usar este metodo.
    def sign(self,msg):
        if self.privK:
            d = bytestobits(self.hash_f(msg))
            return [ self.privK[j][d[j]] for j in range(self.n) ]
        else:
            raise Exception("No hay clave privada; use generate_keys() para generar un par de claves!")
    
    # Devuelve True si el mensaje es autentico, False si no lo es. Es necesario haber inicializado esta clase con la clave publica del emisor.
    def verify(self,msg,signature):
        if self.pubK:
            d = bytestobits(self.hash_f(msg))
            
            for j in range(self.n):
                if self.ow_f(signature[j]) != self.pubK[j][d[j]]: return False
            
            return True
        else:
            raise Exception("No hay clave publica!")
\end{lstlisting}

\section{Algoritmo de b�squeda por fuerza bruta}
\begin{lstlisting}[language=Python]
start = time.time() # Para medir el tiempo de ejecucion

k=1
for msg in msgGen.iterator(): # msgGen es el generador de mensajes
    digest = lamport.bytestobits(sha256(msg.encode('utf-8')))
    if (digest == dig1):
        print("Documento falsificado!")
        print(msg)
        print("Comprobacion: " + str(l.verify(msg.encode('utf-8'),sig1)))
        break
    k += 1

print(str(k)+" documentos probados.")
print("%s segundos" % (time.time() - start))
\end{lstlisting}

\section{Algoritmo de b�squeda con 2 firmas}
\begin{lstlisting}[language=Python]
start = time.time() # Para medir el tiempo de ejecucion

k=1
for msg in msgGen.iterator(): # msgGen es el generador de mensajes
    digest = lamport.bytestobits(sha256(msg.encode('utf-8')))
    mysig = []
    for k in range(len(digest)):
        if digest[k]==dig1[k]:
            mysig.append(sig1[k])
        elif digest[k]==dig2[k]:
            mysig.append(sig2[k])
        else:
            break
    if len(mysig)==len(sig1):
        print("Documento falsificado!")
        print(msg)
        print("Comprobacion: " + str(l.verify(msg.encode('utf-8'),mysig)))
        break
    cont += 1

print(str(cont)+" documentos probados.")
print("%s segundos" % (time.time() - start))
\end{lstlisting}