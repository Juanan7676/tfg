\def\K{\mathbb{K}}

\chapter{Productos tensoriales}
\label{apA}

\section{Definiciones y propiedades b�sicas}
El producto tensorial de dos espacios vectoriales puede definirse de diversas maneras. Por ejemplo, siguiendo \cite{algebra}, una definici�n es la siguiente:

\bdefinicion
Dados $U,V$ dos espacios vectoriales de dimensi�n finita sobre un mismo cuerpo $\K$, se define el producto tensor, y denotado $U \otimes V$, al espacio dual del espacio vectorial formado por todas las formas bilineales en $U \oplus V$.

Para cada $x\in U$ e $y\in V$, el producto tensorial $z = x \otimes y \in U \otimes V$ est� definido como la aplicaci�n tal que $z(w) = w(x,y)$ para cada forma bilineal $w$ de $U\oplus V$.
\edefinicion

A partir de esta definici�n, es f�cil ver las siguientes propiedades:

\bproposicion

Con la notaci�n anterior, para cada $x_1,x_2\in U$, $y_1,y_2\in V$, $\lambda\in \K$, se verifica que:
\begin{enumerate}
	\item $ (x_1+x_2) \otimes y_1 = x_1 \otimes y_1 + x_2 \otimes y_1 $.
	\item $ x_1 \otimes (y_1+y_2) = x_1 \otimes y_1 + x_1 \otimes y_2 $.
	\item $ \lambda (x_1 \otimes y_1) = (\lambda x_1) \otimes y_1 = x_1 \otimes (\lambda y_1) $.
\end{enumerate}
\eproposicion
\begin{proof}
	\phantom{a} \\
	\begin{enumerate}
		\item Denotemos $z = (x_1+x_2) \otimes y_1$. Por definici�n, para cada forma bilineal $w$, se verifica que $ z(w) = w(x_1+x_2,y_1)$. Aplicando bilinealidad en $w$, entonces $z(w) = w(x_1,y_1) + w(x_2,y_2)$. Denotando ahora $z_1 = x_1 \otimes y_1$ y $z_2 = x_2 \otimes y_1$, entonces tenemos que $z(w) = z_1(w)+z_2(w)$. Como esto se verifica para cada forma bilineal $w$, entonces $z = z_1+z_2$, como quer�amos ver.
		\item Esta igualdad es an�loga al caso anterior, aplicando bilinealidad en la segunda componente.
		\item Denotando $z = (\lambda x_1) \otimes y_1, z' = x_1 \otimes y_1$, entonces para cada forma bilineal $w$, se tiene que $z(w) = w(\lambda x_1,y_1) = \lambda w(x_1,y_1) = \lambda z'(w)$. A partir de aqu�, $\lambda z' = z$, como quer�amos ver. An�logamente se puede ver, aplicando bilinealidad en la segunda componente, que $\lambda(x_1 \otimes y_1) = x_1 \otimes (\lambda y_1)$, lo cual concluye la prueba.
	\end{enumerate}
\end{proof}

Nuestro objetivo, ahora, es determinar una base algebraica de $U \otimes V$ (y con ello, su dimensi�n). Para ello, necesitamos dos lemas previos sobre formas bilineales (\cite{algebra}).

\blema
Sea $U$ un espacio vectorial de dimensi�n $n$ sobre un cuerpo $\K$ con base $\{ u_1,\dots,u_n \}$, y $V$ un espacio vectorial de dimensi�n $m$ sobre el mismo cuerpo $\K$, con base $\{ v_1,\dots,v_m \}$. Si $\{ \alpha_{ij} \}$ es cualquier conjunto de escalares, $i=1,\dots,n$, $j = 1,\dots,m$, entonces existe una �nica forma bilineal $w$ en $U\oplus V$ tal que $w(u_i,v_j)=\alpha_{ij}$ para cada $i,j$.
\elema
\begin{proof}
	Sean $u\in U$, $u = \sum_{i=1}^n \alpha_i u_i$, $v\in V$, $v = \sum_{j=1}^m \beta_j v_j$. Una forma bilineal $w$ en $U\otimes V$ verifica que $w(u_i,v_j)=\alpha_{ij}$ si y s�lo si:
	
	\[ w(u,v) = \sum_{i=1}^n \sum_{j=1}^m \alpha_i \beta_j w(u_i,v_j) = \sum_{i=1}^n \sum_{j=1}^m \alpha_i \beta_j \alpha_{ij}, \quad \mbox{ para cada } u\in U, v\in V. \]
	Definiendo $w$ a partir de esta igualdad se hace clara la existencia, y viendo que esta definci�n es una condici�n necesaria para que $w$ verifique la propiedad expuesta, se tiene la unicidad.
\end{proof}

\blema
\label{lemaA2}
Sea $U$ un espacio vectorial de dimensi�n $n$ sobre un cuerpo $\K$ con base $\{ u_1,\dots,u_n \}$, y $V$ un espacio vectorial de dimensi�n $m$ sobre el mismo cuerpo $\K$, con base $\{ v_1,\dots,v_m \}$. Entonces, una base del espacio vectorial de todas las formas bilineales sobre $U \oplus V$ es $\{ w_{pq} \} $, $p=1,\dots,n$, $q=1\dots,m$, verificando que $ w_{pq}(u_i,v_j)=\delta_{ip}\delta_{qj} $ (donde $\delta_{ij}$ representa la delta de Kronecker, que vale 1 si $i=j$ y 0 en caso contrario).
\elema
\begin{proof}
	La existencia de dichos $w_{pq}$ se deduce del lema anterior. Estos elementos son linealmente independientes, puesto que
	\[ \sum_{p=1}^n \sum_{q=1}^m \alpha_{pq}w_{pq} = 0 \]
	Implica que, evaluando en $(u_i,v_j)$ para $i=1,\dots,n$,$j=1,\dots,m$:
	\[ 0 = \sum_{p=1}^{n} \sum_{q=1}^m \alpha_{pq}\delta_{ip}\delta_{jq} = \alpha_{ij} \]
	Sea ahora $w$ una forma bilineal arbitraria de $U\oplus V$, y denotemos por $\alpha_{ij} = w(u_i,v_j)$, $i=1,\dots,n$,$j=1,\dots,m$. Veremos que $w = \sum_{p=1}^n \sum_{q=1}^m \alpha_{pq} w_{pq}$. En efecto, si $u = \sum_{i=1}^n \alpha_i u_i \in U$, $v = \sum_{j=1}^m \beta_j v_j \in V$, entonces
	\[ w_{pq}(u,v) = \sum_{i=1}^n \sum_{j=1}^m \alpha_i\beta_j\delta_{ip}\delta_{jq} = \alpha_p\beta_q \]
	Y por tanto,
	\[ w(x,y) = \sum_{i=1}^n \sum_{j=1}^m \alpha_i\beta_j\alpha_{ij} =\sum_{i=1}^n \sum_{j=1}^m \alpha_i\beta_j w_{pq}(x,y) \]
\end{proof}

Esto prueba que la dimensi�n del conjunto de todas las formas bilineales de $U\oplus V$ es $mn$, el cardinal de la base hallada. De aqu� es inmediato que el producto tensorial de dos espacios vectoriales $U$ y $V$ tambi�n tiene dimensi�n $mn$, puesto que es el dual del espacio anterior. Podemos precisar y dar una base de $U \otimes V$.

\bteoremaa
Si $\{ x_1,\dots,x_n \}$, $\{ y_1,\dots,y_m \}$ son bases de dos espacios vectoriales $U$ y $V$ respectivamente, entonces el conjunto $ \{ z_{ij} = x_i \otimes y_j \} $, $i=1,\dots,n$, $j=1,\dots,m$ es una base de $U \otimes V$.
\eteorema
\begin{proof}
	Consideremos la base $\{ w_{pq} \} $ del espacio de las formas bilineales en $U \oplus V$, con $p=1,\dots,n$, $q=1\dots,m$ definida como en el Lema \ref{lemaA2}. Consideremos su base dual $ \{ w_{pq}' \} $, de manera que $ w_{pq}'(w_{ij}) = \delta_{ip}\delta_{iq} $ (es decir, vale $1$ si $i=p$,$j=q$ y $0$ en cualquier otro caso). Si $w = \sum_{p=1}^n \sum_{q=1}^m \alpha_{pq} w_{pq}$ es una forma bilineal arbitraria en $U\oplus V$, entonces
	
	\[ w'_{ij}(w) = \sum_{p=1}^n \sum_{q=1}^m \alpha_{pq}w'_{ij}(w_{pq}) = \sum_{p=1}^n \sum_{q=1}^m \alpha_{pq}\delta_{ip}\delta_{iq} = \alpha_{ij} = w(x_i,y_j) = z_{ij}(w)  \]
	
	De aqu�, $z_{ij} = w_{ij}'$ y puesto que $w_{ij}'$ forma una base de $U \otimes V$ por construcci�n, tambi�n los elementos $x_i \otimes y_j$.
\end{proof}

\section{Aplicaciones lineales entre productos tensoriales de espacios}
Fijemos dos espacios vectoriales $U,V$. �Qu� forma tienen las aplicaciones lineales $f : U \otimes V \to U \otimes V$? Sean $x\in U$, $y \in V$ y $g : U \to U$, $h : V \to V$ dos aplicaciones lineales. Entonces, de manera natural, podemos definir una aplicaci�n $g \otimes h : U \otimes V \to U \otimes V$ como
\begin{equation} \label{linealtensor} (g \otimes h)(x\otimes y) = g(x) \otimes h(y) \end{equation}

A partir de la linealidad del producto tensorial y de $A$ y $B$, es trivial ver que $A \otimes B$ es lineal. La definici�n de esta aplicaci�n es an�loga si las aplicaciones lineales $A$ y $B$ no son endomorfismos.

\bteoremaa
Con la notaci�n anterior, si $A$ y $B$ son las matrices asociadas de $g$ y $h$ respectivamente, entonces la matriz asociada al operador $g\otimes h$ viene dada por la matriz

\[ A\otimes B = \begin{pmatrix}
A_{11} B & A_{12} B & \dots & A_{1n} B \\
A_{21} B & A_{22} B & \dots & A_{2n} B \\
\vdots & \vdots & \ddots & \vdots \\
A_{m1} B & A_{m2} B & \dots & A_{mn} B
\end{pmatrix} \]
\eteorema

En la notaci�n anterior, $A_{ij}B$ denota el producto de la entrada $(i,j)$ de la matriz $A$ por la matriz $B$. La matriz $A \otimes B$ est� definida, como puede verse, por bloques; si $A$ es una matriz $m\times n$ y $B$ es una matriz de dimensi�n $p\times q$, $A\otimes B$ es una matriz de dimensi�n $mp \times nq$. A este producto $\otimes$ entre matrices se le denomina \textit{producto de Kronecker}. La demostraci�n de este teorema puede hacerse a partir de la ecuaci�n \eqref{linealtensor}, evaluando $g\otimes h$ por los elementos de la base de $U \otimes V$.