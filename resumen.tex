
\begin{resumen}

% Pon tu resumen aqu� en espa�ol.

Se ha creado e investigado un modelo de interacci�n entre un fabricante y el estado  donde el fabricante produce un solo producto y el estado controla el nivel de  contaminaci�n. Se considera una econom�a local con un problema de contaminaci�n  almacenada, que debe escoger entre inversiones en producci�n y medio ambiente  (funciones de control). El modelo es descrito por un sistema de dos ecuaciones  diferenciales con dos controles acotados. La mejor estrategia de control se encuentra  anal�ticamente usando el Principio del M�ximo de Pontryagin y el Teorema de Green.

En este trabajo analizamos algunos aspectos de la Teor�a de Control que incluyen consideraciones sobre sus or�genes, sus motivaciones y su evoluci�n. Describimos algunos
elementos matem�ticos fundamentales y diversos avances que se caracterizan a la vez por su
inter�s cient�fico y su transcendencia desde un punto de vista social, tecnol�gico e industrial. Tambi�n, mencionamos algunos de los retos que se plantean en esta disciplina para un futuro
inmediato.

Hay dos divisiones principales de la teor�a de control, es decir, cl�sicos y  modernos, que tienen implicaciones directas sobre las aplicaciones de ingenier�a de control. La teor�a desarrollada para el control de procesos, desde el punto cl�sico y moderno tiene su base esencial en el conocimiento de la din�mica del proceso que se desea controlar.Desde la teor�a cl�sica de control, considerando el caso m�s sencillo de un sistema lineal de una entrada y una salida (SISO) del dise�o del sistema. Estadin�mica normalmente se expresa asiendo uso de ecuaciones diferenciales ordinarias, y en el caso de sistemas lineales, usando de igual manera la transformada de Laplace para obtener as� de una representaci�n matem�tica que relaciona la se�al que se quiere controlar y la se�al de entrada al sistemas.Un controlador dise�ado por la teor�a cl�sica por lo general requiere en ellugar de sinton�a debido a las aproximaciones de dise�o. Los controladores dise�ado con la teor�a de control cl�sica comunes son los CONTROLADORESPID.En contraste, la teor�a de control moderna se lleva acabo estrictamente enel complejo-s o el dominio de la frecuencia y puede lidiar con m�ltiples entradas y m�ltiples salidas (MIMO) de sistemas. Esto para dise�o sofisticado, como el control de aviones de combate etc.,. En el dise�o moderno, un sistema representa como un conjunto de primer orden ecuaciones diferenciales. El �rea de control moderno tiene muchas �reas que explorar. Lo cual se detallara en este trabajo.

\end{resumen}

