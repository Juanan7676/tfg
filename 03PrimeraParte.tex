\chapter{Computaci�n cu�ntica}

\section{Introducci�n}
El objetivo de este cap�tulo es describir el (principal) algoritmo que pone en peligro toda la criptograf�a cl�sica que se usa actualmente en gran medida: el algoritmo de Shor, que permite factorizar cualquier n�mero natural en $O((\log N)^3)$ pasos (\cite{quantuminformation}, p�g. 233), donde $N$ es el n�mero a factorizar. El algoritmo contiene dos partes, una parte cl�sica que se ejecutar�a en un ordenador cl�sico, y una parte cu�ntica que se ejecutar�a en un ordenador cu�ntico. La idea clave del algoritmo subyace en el siguiente resultado.

\bproposicion
\label{prop1}
Sea $N$ un n�mero compuesto de $L$ bits, y sea $x\in \mathbb{Z}_N$ con \\ $1<x<N-1$ una soluci�n de la ecuaci�n
\begin{equation} \label{eqcuadrados} x^2 = 1\mod N \end{equation}
Entonces, o bien $\mcd(x-1,N)$ o bien $\mcd(x+1,N)$ es un factor no trivial de $N$ y se puede calcular en $O(L^3)$ operaciones.
\eproposicion
\begin{proof}
	Ya que $x^2 - 1 = 0 \mod N$, entonces $N|_{(x^2+1)=(x+1)(x-1)}$. De aqu�, $N$ debe tener un factor com�n con $(x+1)$ � con $(x-1)$, es decir, $\mcd(x+1,N)>1$ � $\mcd(x-1,N)>1$. Adem�s, ya que $1 < x < N-1$, entonces $2 < x+1 < N$ y $0 < x-1 < N-2$, en cualquier caso, $x-1 < N$ y $x+1 < N$ y por tanto $\mcd(x-1,N) < N$ y $\mcd(x+1,N) < N$. Esto prueba que ninguno puede ser un factor trivial de $N$. Mediante el algoritmo de Euclides, estos factores pueden calcularse en $O(L^3)$ operaciones (ver \cite{quantuminformation}, p�g. 629).
\end{proof}

En general, encontrar una soluci�n de \eqref{eqcuadrados} es dif�cil. Sin embargo, existe una estrategia para abordar este problema. Si $1 < y < N-1$ es cualquier n�mero coprimo con $N$ y resulta que existe $r\in\mathbb{N}$ \textit{par} tal que $y^r = 1 \mod N$, entonces $y^{r/2}$ ser�a una soluci�n de \eqref{eqcuadrados}; y adem�s cumplir�a las hip�tesis de la Proposici�n \ref{prop1} si $ y^{r/2} \neq \pm 1 \mod N $. Ese n�mero $r$ es el \textit{orden} del elemento $y$ dentro del grupo multiplicativo $\mathbb{Z}/N\mathbb{Z}$. Resulta que si escogemos este n�mero $y$ aleatoriamente entre $2$ y $N-2$ (si no es coprimo con $N$, entonces habr�amos encontrado un factor calculando $\mcd(y,N)$), entonces es muy probable que verifique las condiciones expuestas, lo cual nos permitir�a calcular los factores como enuncia la Proposici�n \ref{prop1}. Este hecho se recoge en este resultado.

\bproposicion
Sea $N = p_1^{\alpha_1}p_2^{\alpha_2}\cdots p_m^{\alpha_m}$ una factorizaci�n en n�meros primos de un n�mero impar. Sea $x$ un elemento escogido aleatoriamente de $(\mathbb{Z}/N\mathbb{Z})^*$, y sea $r$ el orden de $x$ m�dulo $N$. Entonces,
\[ P[\mbox{r es par y que }x^{r/2}\neq -1 \mod N]\geq 1-\frac{1}{2^m} \]
\eproposicion